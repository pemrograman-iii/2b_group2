\section{Fathi Rabbani / 1164074}
\subsection{Teori}
\begin{enumerate}
\item Sejarah dan Penjelasan CSV
\subitem Penggunaan dari format file CSV itu sendiri untuk memudahkan pembuatan data dengan menggunakan tanda koma sebagai pembatas dari datanya agar mudah untuk dibaca pada kolom.
\subitem CSV sendiri dibuat untuk dapat menangani pembuatan sejumlah data yang berukuran besar, mempermudah program dalam membaca datanya kedalam kolom - kolom. seperti contoh dalam membacanya menggunakan aplikasi Excel sehingga mempermudah dalam proses import dan eksport datanya. csv sendiri sudah ada pada tahun 1972 dengan pengembangnya adalah IBM namun penggunaannya masuk pada tahun 1983 yang berbarengan dengan adanya SuperCalc spreadsheet.

\item Aplikasi CSV
\begin{itemize}
\item Microsoft Excel
\item Open Office Calc
\item Google Docs
\item Libre Office
\item Apache Open Office
\end{itemize}

\item Menulis dan Membaca csv di Excel atau Spreadsheet
\subitem Menulis, cara menuliskan csv adalah dengan menggunakan tanda baca koma pada bagian data yang ingin dipisah contohnya \lstinputlisting[firstline=8, lastline=29]{src/chapter4/coba.csv}
\subitem Membaca, file csv dapat dibaca pada program aplikasi Excel dengan menampilkan hasil data dari setiap data yang dipisah dengan tanda baca koma menjadi kolom - kolom hasilnya ada pada \ref{fig1}

\item Library CSV
\subitem CSV atau comma separated value adalah salah satu tipe file yang digunakan secara luas di dunia programming. Tidak hanya itu CSV pun sering digunakan dalam pengolahan informasi yang dihasilkan spreadsheet untuk diproses lebih lanjut melalui mesin analitik. CSV pun dianggap sebagai file yang agnostik karena dapat digunakan oleh berbagai database untuk proses backup data.

\item Library Pandas
\subitem pandas merupakan library pada pemrograman python yang berguna untuk mengolah dan meyediakan struktur data serta analisa data yang mudah untuk dibaca dan dipahami seperti pada struktur data tabel. pandas dapat melakukan proses perbandingan data, penggabungan dataset, penanganan dataset yang hilang dll. pandas dapat juga digunakan sebagai pemrosesan data Statistik dengan pembacaan datanya menggunakan struktur Spreadsheet.

\item Fungsi pada Library CSV
\begin{itemize}
\item Menulis data CSV
\lstinputlisting[firstline=8, lastline=29]{src/chapter4/coba.csv}
\item Hasil dari menullis data CSV
\lstinputlisting[firstline=31, lastline=37]{src/chapter4/coba.csv}
\item Membaca data CSV
\lstinputlisting[firstline=40, lastline=52]{src/chapter4/coba.csv}
\item Hasil pembacaan data CSV
\lstinputlisting[firstline=54, lastline=60]{src/chapter4/coba.csv}
\end{itemize}

\item Fungsi pada Library Pandas
\begin{itemize}
\item Kode
\lstinputlisting[firstline=62, lastline=67]{src/chapter4/coba.csv}
\item Hasil
\lstinputlisting[firstline=69, lastline=73]{src/chapter4/coba.csv}
\end{itemize}
\end{enumerate}

\begin{figure}[!htbp]
	\centering
	\includegraphics[width=1\textwidth]{figures/chapter4/1164074/1}
	\caption{hasil csv pada Excel}
	\label{fig1}
\end{figure}

\subsection{Praktek}
\begin{enumerate}
\item Fungsi Library CSV List
\lstinputlisting[firstline=1, lastline=8]{src/chapter4/1164074/c4_1164074_csv.py}
\item Fungsi Library CSV Dictionary
\lstinputlisting[firstline=10, lastline=16]{src/chapter4/1164074/c4_1164074_csv.py}
\item Fungsi Library pandas List
\lstinputlisting[firstline=1, lastline=4]{src/chapter4/1164074/c4_1164074_pandas.py}
\item Fungsi Library pandas Dictionary
\lstinputlisting[firstline=5, lastline=7]{src/chapter4/1164074/c4_1164074_pandas.py}
\item Fungsi Library pandas Date to DataFrame
\lstinputlisting[firstline=9, lastline=11]{src/chapter4/1164074/c4_1164074_pandas.py}
\item Fungsi Library pandas Index Colomn
\lstinputlisting[firstline=13, lastline=15]{src/chapter4/1164074/c4_1164074_pandas.py}
\item Fungsi Library pandas Change Att Colomn
\lstinputlisting[firstline=17, lastline=19]{src/chapter4/1164074/c4_1164074_pandas.py}
\item Fungsi Library CSV main.py
\lstinputlisting[firstline=1, lastline=7]{src/chapter4/1164074/main.py}
\item Fungsi Library pandas main2.py
\lstinputlisting[firstline=1, lastline=7]{src/chapter4/1164074/main2.py}
\end{enumerate}

\subsection{Error}
\begin{itemize}
\item Error Code
\subitem
ValueError: Date column tanggal already in dict
\item Keterangan Error
\subitem
data sudah berupa dictoinary sehingga tidak memerlukan pemecahan lagi dengan code berikut 
\lstinputlisting[firstline=5, lastline=6]{src/chapter4/1164074/1164074_Error.py}
\item Solusi
\subitem
\lstinputlisting[firstline=12, lastline=13]{src/chapter4/1164074/1164074_Error.py}
\end{itemize}

\section {Kevin Natanael Nainggolan 1174059}
	\begin {enumerate}
		\item Apa itu fungsi csv, jelaskan sejarah dan contohnya 
			\lstinputlisting [firstline=10, lastline=14]{src/teoric4.py}
		\item Aplikasi-aplikasi apa saja yang bisa menciptakan file csv? 
			\lstinputlisting [firstline=18, lastline=22]{src/teoric4.py}
		\item Jelaskan bagaimana cara menulis dan membaca file csv di excel atau spreadsheet
			\lstinputlisting [firstline=26, lastline=39]{src/teoric4.py}
		\item Jelaskan sejarah library csv
			\lstinputlisting [firstline=43, lastline=50]{src/teoric4.py}
		\item Jelaskan sejarah library pandas
			\lstinputlisting [firstline=54, lastline=60]{src/teoric4.py}
		\item Jelaskan fungsi-fungsi yang terdapat di library csv
			\lstinputlisting [firstline=64, lastline=68]{src/teoric4.py}
		\item Jelaskan fungsi-fungsi yang terdapat di library pandas
			\lstinputlisting [firstline=72, lastline=75]{src/teoric4.py}
	\end {enumerate}




\section{Yusniar Nur Syarif Sidiq/1164089}
\subsection{Pemahaman Teori}

\begin{enumerate}

\item Apa itu fungsi file csv, jelaskan sejarah dan contoh.
	\subitem File csv merupakan jenis file khusus yang dapat kita buat dan edit di dalam Excel. File csv akan menyimpan informasi data yang dipisahkan dengan koma atau tanda titik koma, dimana artinya file csv tidak menyimpan data dalam bentuk kolom. Saat pertama kali rilis, excel menggunakan format file dalam bentuk biner yaitu BIFF sebagai format file utama. Namun setelah Microsoft merilis Ofice System 2007, Excel telah menggantikan format utamanya menjadi XML. Meskipun mendukung format XML baru, Excel masih mendukung format BIFF, tidak hanya itu excel juga telah mendukung format CSV, DBF, SYLK, DIF, dan format-format lainnya. Fungsi dari file csv itu sendiri adalah mempermudah dalam pencarian data dan pengimputan data ke dalam database sederhana. File csv mulai digunakan pada tahun 1983 akan tetapi format file csv sudah ada dari tahun 1972. Contoh file dengan format csv dapat dilihat pada figure \ref{YNCSV1}

	\begin{figure}[ht]
		\centering{\includegraphics[scale=0.5]{figures/chapter4/YN/Chapter4/YNCSV1.png}}
		\caption{Contoh File CSV}
		\label{YNCSV1}
	\end{figure}

\item Aplikasi - aplikasi apa saja yang bisa menciptakan file csv.
	\subitem Untuk membuat file dengan format CSV, kita dapat menggunakan software bawaan Microfsoft Ofice yaitu Excel. Bukan hanya Microsoft Excel, kita juga dapat membuat file CSV dengan bantuan text editor. Jika kita ingin membuat file csv secara online dapat menggunakan Google Spreadshare. Apabila OS PC kita menggunakan Linux dapat menggunakan LibreOfficecalc.

\item Jelaskan bagaimana cara menulis dan membaca file csv di excel atau spreadsheet.
	\subitem Cara membuat file CSV dengan Excel
			\begin{itemize}
				\item Buka software Microsoft Excel
				\item Pilih new document
				\item Buatlah judul kolom yang ingin kita rekam
				\item Isikan informasi - informasi pada setiap kolom
				\item Simpan dengan menggunakan metode save as
				\item Cari dan pilih format csv
				\item Pilih button save untuk melakukan penyimpanan
			\end{itemize}
	\subitem Cara membaca file CSV dengan Excel
			\begin{itemize}
				\item Buka software Microsoft Excel
				\item Lakukan perintah open file
				\item Cari file csv yang sudah kita buat sebelumnya
				\item Pilih button open untuk membaca file csv pada Microsoft Excel
			\end{itemize}
	\subitem Cara membaca file csv dari Excel
		\lstinputlisting[firstline=1, lastline=9]{src/chapter4/1164089.py}
	\subitem Cara membuat file csv
		\lstinputlisting[firstline=12, lastline=16]{src/chapter4/1164089.py}

\item Jelaskan sejarah library csv. 
	\subitem Pada tahun 1972 adalah terbentuknya format file csv namun bukan hanya itu saja, pada saat itupun terbentuk juga yang namanya library pandas.Seiring dengan lahirnya bahasa pemrograman python, library mulai dibuat dan dikembangkan oleh Kevin Altis. Dengan kata lain CSV dibentuk pada tahun 1972 dan sudah satu paket baik dalam librarynya maupun format filenya. 

\item Jelaskan sejarah library pandas.
	\subitem Developer yang bernama Wes McKinney telah mengajarkan pandas pada tahun 2008 ketika ia berada di AQR Capital Management, karena kebutuhan akan alat kinerja tinggi yang fleksibel untuk melakukan analisis kuantitatif pada data keuangan. Sebelum meninggalkan AQR, dia dapat meyakinkan manajemen untuk mengizinkan membuka sumber library. Pegawai AQR lainnya yaitu Chang She, telah bergabung dengan upaya ini pada 2012 sebagai kontributor utama kedua ke library. Pada tahun 2015, pandas telah menandatangani sebagai proyek NumFocus yang disponsori secara fiskal. Pada saat itulah Library Pandas mulai berjalan dan digunakan.

\item Jelaskan fungsi-fungsi yang terdapat di library csv.
	\subitem Ada dua fungsi pada library csv, yaitu csv.reade dan csv.writer. Dimana fungsi tersebut memiliki tugas yang berbeda-beda. Untuk csv.reader bertugas sebagai membaca file csv sedangkan csv.writer bertugas membuat file csv.

\item Jelaskan fungsi-fungsi yang terdapat di library pandas.
	\subitem Untuk library pandas sama saja dengan library csv namun bedanya hanya cara penulisan source codenya saja. Untuk membaca file csv pada library pandas membutuhkan perintah pandas.read\_csv sedangkan untuk membuat file csv membutuhkan perintah pandas.write\_csv.

\end{enumerate}

\subsection{Keterampilan Pemrograman}
\begin{enumerate}
	\item Membaca file csv pada lib csv dengan mode list

		\lstinputlisting[firstline=1, lastline=9]{src/chapter4/1164089/1164089_csv.py}

	\item Membaca file csv pada lib csv dengan mode dictionary

		\lstinputlisting[firstline=11, lastline=16]{src/chapter4/1164089/1164089_csv.py}

	\item Membaca file csv pada lib pandas dengan mode list
		
		\lstinputlisting[firstline=1, lastline=7]{src/chapter4/1164089/1164089_pandas.py}

	\item Membaca file csv pada lib pandas dengan mode dictionary

		\lstinputlisting[firstline=9, lastline=13]{src/chapter4/1164089/1164089_pandas.py}

	\item Mengubah format tanggal menjadi standar DataFrame

		\lstinputlisting[firstline=15, lastline=18	]{src/chapter4/1164089/1164089_pandas.py}
	
	\item Mengubah index kolom

		\lstinputlisting[firstline=21, lastline=25]{src/chapter4/1164089/1164089_pandas.py}

	\item Mengubah atribut atau nama kolom

		\lstinputlisting[firstline=26, lastline=30]{src/chapter4/1164089/1164089_pandas.py}

	\item Membuat program NPM\_main.py dan isikan bagaimana cara membaca file csv dan membuat file csv

		\lstinputlisting[firstline=1, lastline=6]{src/chapter4/1164089/1164089_main.py}
	
	\item Membuat program NPM\_main2.py dan isikan bagaimana cara membaca file csv dan membuat file csv dengan lib pandas

		\lstinputlisting[firstline=1, lastline=6]{src/chapter4/1164089/1164089_main2.py}
	

\end{enumerate}

\subsection{Penaganan Error}

\begin{enumerate}

	\item Tuiskan peringatan error yang di dapat dari mengerjakan praktek ketiga ini, dan jelaskan cara penanganan error tersebut. Buatlah fungsi try except untuk mengulangi error tersebut.

	\subitem Dimana error yang saya dapat merupakan atribute error yaitu dimana kondisi penulisan source code atribute salah atau tidak ditemukan, untuk lebih jelasnya dapat dilihat pada figure \ref{YNC4-3}

	\begin{figure}[ht]
		\centering{\includegraphics[scale=0.5]{figures/chapter4/YN/Chapter4/YNC4-3.png}}
		\caption{Atribute Error}
		\label{YNC4-3}
	\end{figure}

Cara penganannya yaitu ubah atribute yang terdapat pada file python sama dengan yang kita buat di file sebelumnya. Coontohnya dapat dilihat pada figure \ref{YNC4-4}

	\begin{figure}[ht]
		\centering{\includegraphics[scale=0.5]{figures/chapter4/YN/Chapter4/YNC4-4.png}}
		\caption{Atribute Error}
		\label{YNC4-4}
	\end{figure}

Dimana penulisannya harus sama persis dikarenakan Python bersifat case sensitif.

	\subitem Untuk penulisan Try Ecept dapat dilihat pada source code dibawah ini,

	\lstinputlisting[firstline=42, lastline=47]{src/chapter4/1164089/1164089_pandas.py}


\end{enumerate}

\section{Alit Fajar Kurniawan  1174057}
	\subsection{Pemahaman Teori}
		\begin{enumerate}
			\item 
			\begin{itemize}
					\item Fungsi : File csv berfungsi melakukan pencarian data agar menjadi lebih mudah dan cepat, dan juga mempermudah penginputan 
					data ke dalam database secara lebih sederhana.
					\item Sejarah : Pada 10 tahun yang lalu File csv muncul pertama kali sebelum Personal Computer pertama  di dunia sejak 
					sekitar tahun 1972, akan tetapi sebutan file csv digunakan pertama kali pada tahun 1983.
					\item Contohnya : Anda dapat mengekspor kontak dari Google ke dalam file CSV, kemudian mengimpornya ke Outlook.
			\end{itemize}
			
			\item Ada banyak aplikasi yang dapat membuat file berformat CSV, diantaranya adalah :
				    Pada Windows
					\begin{itemize}
						\item Microsoft Excel 2013
						\item Microsoft Works
						\item CCorel Quattro Pro
						\item Apache OpenOffice
						\item LibreOffice
						\item Microsoft Notepad
						\item Intuit Quicken 2015
						\item GenScriber
					\end{itemize}
					Pada Linux
					\begin{itemize}
						\item Apache OpenOffice
						\item LibreOffice
						\item GenScriber
					\end{itemize}
					Pada Mac OS
					\begin{itemize}
						\item Microsoft Excel 2011
						\item Planamesa NeoOffice
						\item Apache OpenOffice
						\item LibreOffice
						\item GenScriber
					\end{itemize}
					
			\item Jelaskan bagaimana cara menulis dan membaca file csv di excel atau spreadsheet?
				\begin{itemize}
					\item Untuk menulis file csv harus berupa baris dan kolom atau bisa juga di sebut berupa tabel.
					\item Untuk membacanya file csv dipisahkannya menggunakan koma atau titik koma.
			\end{itemize}
			
			\item sejarah library csv : Library csv menyediakan fungsionalitas untuk membaca dan menulis ke file CSV. Dirancang untuk bekerja di luar kotak dengan file CSV 
			yang dihasilkan Excel, memudahkan untuk bekerja dengan berbagai format CSV. Library csv berisi objek dan kode lain untuk membaca, menulis, 
			dan memproses data ke file CSV.
			
			\item Sejarah library pandas : Tahun 2008, pengembangan pandas dimulai oleh AQR Capital Management. Pada akhir tahun 2009 pandas menjadi Open Sourced, 
			dimana disupport oleh banyak komunitas atau individu di dunia untuk mengembangkan pandas. Sejak tahun 2015, 
			pandas menjadi NumFOCUS proyek sponsor, ini juga membantu suksesnya pengembangan dari pandas itu sendiri. 
			pandas merupakan struktur data dan data analysis tools untuk bahasa pemrograman Python, 
			dan merupakan BSD-licensed library yang menjadikannya memiliki performa yang tinggi.
			
			\item Jelaskan fungsi-fungsi yang terdapat di library csv?
			Terdapat 2 fungsi yang bisa digunakan oleh library csv
			Pertama,fungsi membaca file csv.
			fungsi ini bisa menggunakan list dan dictionary
			Dengan list :
			\lstinputlisting[firstline=11, lastline=21]{src/chapter4/1174057csvpandas.py}
			Dengan dictionary :
			\lstinputlisting[firstline=24, lastline=33]{src/chapter4/1174057csvpandas.py}
			Kedua,fungsi menulis file csv.
			\lstinputlisting[firstline=36, lastline=40]{src/chapter4/1174057csvpandas.py}
			
			\item Jelaskan fungsi-fungsi yang terdapat di library pandas
			Hampir sama dengan library csv,tp library pandas penulisannya lebih sederhana dan terlihat lebih rapih dari pada library csv.
			\lstinputlisting[firstline=43, lastline=44]{src/chapter4/1174057csvpandas.py}
    
		\end{enumerate}
		
	\subsection{Keterampilan Pemograman}
		\begin{enumerate}
			\item Buatlah  fungsi  (file  terpisah/library  dengan  nama  NPMcsv.py)  untuk  membuka file csv dengan lib csv mode list.
			\lstinputlisting[caption = Fungsi untuk membuka file CSV dengan lib CSV mode list., firstline=10, lastline=15]{src/Chapter4/1174057/1174057csv.py}
	
			\item Buatlah  fungsi  (file  terpisah/library  dengan  nama  NPMcsv.py)  untuk  membuka file csv dengan lib csv mode dictionary.
			\lstinputlisting[caption =  Fungsi untuk membuka file CSV dengan lib CSV mode dictionary., firstline=17, lastline=22]{src/Chapter4/1174057/1174057csv.py}
			
			\item Buatlah fungsi (file terpisah/library dengan nama NPMpandas.py) untuk membuka file csv dengan lib pandas mode list.	
			\lstinputlisting[caption =  Fungsi untuk membuka file CSV dengan lib Pandas mode list., firstline=10, lastline=13]{src/Chapter4/1174057/1174057pandas.py}
			
			\item Buatlah fungsi (file terpisah/library dengan nama NPMpandas.py) untuk membuka file csv dengan lib pandas mode dictionary.			
			\lstinputlisting[caption =  Fungsi untuk membuka file CSV dengan lib Pandas mode dictionary., firstline=15, lastline=19]{src/Chapter4/1174057/1174057pandas.py}
			
			\item  Buat fungsi baru di NPMpandas.py untuk mengubah format tanggal menjadi standar dataframe.			
			\lstinputlisting[caption =  Fungsi untuk mengubah format tanggal menjadi standar dataframe., firstline=21, lastline=24]{src/Chapter4/1174057/1174057pandas.py}
			
			\item Buat fungsi baru di NPMpandas.py untuk mengubah index kolom.			
			\lstinputlisting[caption =  Fungsi untuk mengubah index kolom., firstline=26, lastline=30]{src/Chapter4/1174057/1174057pandas.py}
			
			\item Buat fungsi baru di NPMpandas.py untuk mengubah atribut atau nama kolom.			
			\lstinputlisting[caption =  Fungsi untuk mengubah atribut atau nama kolom., firstline=32, lastline=36]{src/Chapter4/1174057/1174057pandas.py}
			
			\item Buat program main.py yang menggunakan library NPMcsv.py yang membuat dan membaca file csv.			
			\lstinputlisting[caption =  Membuat dan mebaca file CSV menggunakan library 1174006pandas., firstline=8, lastline=13]{src/Chapter4/1174057/main.py}
			
			\item Buat program main2.py yang menggunakan library NPMpandas.py yang membuat dan membaca file csv.			
			\lstinputlisting[caption = Membuat dan mmebaca file CSV menggunakan library 1174006pandas., firstline=8, lastline=13]{src/Chapter4/1174057/main2.py}
			
		\end{enumerate}
	
	\subsection{Ketrampilan Penanganan Error}
			Error yang di dapatkan dari mengerjakan tugas ini adalah type error, mengatasinya dengan cara mengecheck kembali codingannya
			kemudian run kembali aplikasinya
			berikut contoh Penggunaan fungsi try dan exception

			\lstinputlisting{src/chapter4/1174057/1174057_error.py}
\section {Kevin Natanael Nainggolan 1174059}
	\begin{enumerate}
		\item Apa itu fungsi csv, jelaskan sejarah dan contohnya 
			\lstinputlisting [firstline=10, lastline=14]{src/teoric4.py}
		\item Aplikasi-aplikasi apa saja yang bisa menciptakan file csv? 
			\lstinputlisting [firstline=18, lastline=22]{src/teoric4.py}
		\item Jelaskan bagaimana cara menulis dan membaca file csv di excel atau spreadsheet
			\lstinputlisting [firstline=26, lastline=39]{src/teoric4.py}
		\item Jelaskan sejarah library csv
			\lstinputlisting [firstline=43, lastline=50]{src/teoric4.py}
		\item Jelaskan sejarah library pandas
			\lstinputlisting [firstline=54, lastline=60]{src/teoric4.py}
		\item Jelaskan fungsi-fungsi yang terdapat di library csv
			\lstinputlisting [firstline=64, lastline=68]{src/teoric4.py}
		\item Jelaskan fungsi-fungsi yang terdapat di library pandas
			\lstinputlisting [firstline=72, lastline=75]{src/teoric4.py}
	\end {enumerate}	
	\begin{enumerate}
			\item\lstinputlisting{src/chapter4/1174057/1174057_error.py}	
	\end{enumerate}
			
\section{Muhammad Iqbal Panggabean 1174063}
\subsection{Praktek}
\begin{enumerate}
	\item Buatlah  fungsi  (file  terpisah/library  dengan  nama  NPMcsv.py)  untuk  membuka file csv dengan lib csv mode list.
	
	\lstinputlisting[caption = Fungsi untuk membuka file CSV dengan lib CSV mode list., firstline=10, lastline=15]{src/chapter4/1174063/1174063csv.py}
	
	\item Buatlah  fungsi  (file  terpisah/library  dengan  nama  NPMcsv.py)  untuk  membuka file csv dengan lib csv mode dictionary.
	
	\lstinputlisting[caption =  Fungsi untuk membuka file CSV dengan lib CSV mode dictionary., firstline=17, lastline=22]{src/chapter4/1174063/1174063csv.py}
	
	\item Buatlah fungsi (file terpisah/library dengan nama NPMpandas.py) untuk membuka file csv dengan lib pandas mode list.
	
	\lstinputlisting[caption =  Fungsi untuk membuka file CSV dengan lib Pandas mode list., firstline=10, lastline=13]{src/chapter4/1174063/1174063pandas.py}
	
	\item Buatlah fungsi (file terpisah/library dengan nama NPMpandas.py) untuk membuka file csv dengan lib pandas mode dictionary.
	
	\lstinputlisting[caption =  Fungsi untuk membuka file CSV dengan lib Pandas mode dictionary., firstline=10, lastline=13]{src/chapter4/1174063/1174063pandas.py}
	
	\item  Buat fungsi baru di NPMpandas.py untuk mengubah format tanggal menjadi standar dataframe.
	
	\lstinputlisting[caption =  Fungsi untuk mengubah format tanggal menjadi standar dataframe., firstline=15, lastline=19]{src/chapter4/1174063/1174063pandas.py}
	
	\item Buat fungsi baru di NPMpandas.py untuk mengubah index kolom.
	
	\lstinputlisting[caption =  Fungsi untuk mengubah index kolom., firstline=21, lastline=24]{src/chapter4/1174063/1174063pandas.py}
	
	\item Buat fungsi baru di NPMpandas.py untuk mengubah atribut atau nama kolom.
	
	\lstinputlisting[caption =  Fungsi untuk mengubah atribut atau nama kolom., firstline=26, lastline=30]{src/chapter4/1174063/1174063pandas.py}
	
	\item Buat program main.py yang menggunakan library NPMcsv.py yang membuat dan membaca file csv.
	
	\lstinputlisting[caption =  Membuat dan mebaca file CSV menggunakan library 1174006pandas., firstline=8, lastline=13]{src/chapter4/1174063/1174063main.py}
	
	\item Buat program main2.py yang menggunakan library NPMpandas.py yang membuat dan membaca file csv.
	
	\lstinputlisting[caption = Membuat dan mmebaca file CSV menggunakan library 1174006pandas., firstline=8, lastline=13]{src/chapter4/1174063/1174063main2.py}
\end{enumerate}

\subsection{Ketrampilan Penanganan Error}
Error yang di dapat dari mengerjakan tugas ini adalah type error, cara menaggulaginya dengan cara mengecheck kembali codingannya
kemudian run kembali aplikasinya
berikut contoh Penggunaan fungsi try dan exception
\lstinputlisting{src/chapter4/1174063/1174063_2err.py}

